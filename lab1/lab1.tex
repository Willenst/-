\documentclass[11pt]{article}
\usepackage{amsmath,amssymb,amsthm}
\usepackage{algorithm}
\usepackage[noend]{algpseudocode} 
\usepackage{titlesec}
\usepackage{enumitem}
\usepackage[T2A,T1]{fontenc}
\usepackage{amsmath,amssymb,amsthm}
\usepackage{fancyhdr}
\usepackage{indentfirst}

%---enable russian----

\usepackage[utf8]{inputenc}
\usepackage[russian]{babel}

% PROBABILITY SYMBOLS
\newcommand*\PROB\Pr 
\DeclareMathOperator*{\EXPECT}{\mathbb{E}}


% Sets, Rngs, ets 
\newcommand{\N}{{{\mathbb N}}}
\newcommand{\Z}{{{\mathbb Z}}}
\newcommand{\R}{{{\mathbb R}}}
\newcommand{\Zp}{\ints_p} % Integers modulo p
\newcommand{\Zq}{\ints_q} % Integers modulo q
\newcommand{\Zn}{\ints_N} % Integers modulo N

% Landau 
\newcommand{\bigO}{\mathcal{O}}
\newcommand*{\OLandau}{\bigO}
\newcommand*{\WLandau}{\Omega}
\newcommand*{\xOLandau}{\widetilde{\OLandau}}
\newcommand*{\xWLandau}{\widetilde{\WLandau}}
\newcommand*{\TLandau}{\Theta}
\newcommand*{\xTLandau}{\widetilde{\TLandau}}
\newcommand{\smallo}{o} %technically, an omicron
\newcommand{\softO}{\widetilde{\bigO}}
\newcommand{\wLandau}{\omega}
\newcommand{\negl}{\mathrm{negl}} 

% Misc
\newcommand{\eps}{\varepsilon}
\newcommand{\inprod}[1]{\left\langle #1 \right\rangle}


 
\newcommand{\handout}[5]{
  \noindent
  \begin{center}
  \framebox{
    \vbox{
      \hbox to 5.78in { {\bf Научно-исследовательская практика} \hfill #2 }
      \vspace{4mm}
      \hbox to 5.78in { {\Large \hfill #5  \hfill} }
      \vspace{2mm}
      \hbox to 5.78in { {\em #3 \hfill #4} }
    }
  }
  \end{center}
  \vspace*{4mm}
}

\newcommand{\lecture}[4]{\handout{#1}{#2}{#3}{Scribe: #4}{Система верстки LaTeX}}

\begin{document}

\lecture{}{Лето 2020}{}{Раточка Вячеслав}
\newpage
\renewcommand{\headrulewidth}{0pt}

\section{Конечные цепные дроби}

\textit{Подходящие дроби с нечетными индексами образуют убывающую последовательность:}
\begin{center}
	$C_1>C_3>C_5>...$
\end{center}
\textit{Каждая подходящая дробь с нечетным индексом больше подходящей дроби с четным}
\\
\textit{доказательство:} С помощью теорем 13-7 мы выяснили, что:
\begin{center}
	$C_{k+2}-C_k=(C_{k+2}-C_{k+1})+(C_{k+1}-C_k)=
	(
	\frac{p_{k+2}}{q_{k+2}}-
	\frac{p_{k+1}}{q_{k+1}})+(
	\frac{p_{k+1}}{q_{k+1}}-
	\frac{p_{k}}{q_{k}})=
	\frac{(-1)^{k+1}}{q_{k+2}q_{k+1}}+\frac{(-1)^{k}}{q_{k+1}q_{k}}=
	\frac{(-1)^{k+1}(q_{k+2}-q_k)}{q_{k+2}q_{k+1}q_{k}}
	$
	\end{center}
Отметим, что $q_i>0$ для всех $i\ge0$ а также $q_{k+2}-q_k>0$ по условиям леммы, тогда становится очевидным, что $C_{k+2}-C_k$ имеют тот же знак что и $(-1)^{k}$. Таким образом, если $k$ - четное число, скажем $k=2j$, тогда $C_{2j+2}>C_2j$, откуда: 
\begin{center}
	$C_0<C_2<C_4<...$
\end{center}
Аналогично, если $k$ нечетное число, скажем $k=2j-1$, тогда $C_{2j+1}<C_{2j-1}$, откуда
\begin{center}
	$C_1>C_3>C_5>...$
\end{center}
Остается только показать что любая нечетная подходящая дробь $C_{2j-1}$ будет больше чем любая четная подходящая дробь $C_{2j}$. Имеем $p_{k}q_{k-1}-q_kp_{k-1}=(-1)^{k-1}$, разделив обе части уравнения на $q_kq_{k+1}$, получим 
	\begin{center}
		$C_k-C_{k-1}=\frac{p_k}{q_k}-\frac{p_{k-1}}{q_{k-1}}=\frac{(-1)^{k-1}}{q_kq_{k-1}}$
	\end{center}
Это означает, что $C_{2j}<C_{2j-1}$. Соеденив различные неравенства, получим что:
	\begin{center}
		$CС_{2s}<C_{2s+2r}<C_{2s+2r-1}<C_{2r-1}$
			\end{center}
		Как и предпологалось.
		
		В качестве примере рассмотрим цепную дробь $[2;3,2,5,2,4,2]$. С помощью неспложных вычислений получим следующие подходящие дроби: 
			\begin{center}
				$C_0=2/1,C_1=7/3,C_3=87/38$
				\end{center}
						\begin{center}
				$C_4=190/83,C_5=847/370,C_6=1884/823$
			\end{center}
		
		
		\section{Числа Фибоначи и цепные дроби}
		Ссылаясь на теоремы 13-8, эти подходящие дроби удовлетворяют следующим неравенствам:
		$$2<16/7<190/83<1884/823<847/370<87/38<7/3$$
		В чем легко можно убедиться представив числа в десятичной записи:
		$$2<2.28571...<2.28915...<2.28918...<2.28947...<2.33333...$$
		
	\section{Проблемы}
\noindent
\begin{enumerate}
\item Представьте каждое из приведенных ниже чисел ввиде конечной цепной дроби:
\begin{center}
$(a) -19/17$  $(b) 187/57$  $(c) 71/55$  $(d) 118/303$
			\end{center}
		\item Найдите рациональные числа представленные следующими рациональными дробями:
		\begin{center}
			$(a) [-2;2,4,6,8]$ $(b) [4;2,1,3,1,2,4]$ $(c) [0;1,2,3,4,3,2,1]$
					\end{center}
				\item Если $r=[a_0;a_1,a_2,...,a_n]$, где $r>0$, покажите, что:
				 $$1/r=[0;a_0,a_1,...,a_n].$$
				 \item Представьте следующие цепные дроби в таком же виде, но с нечетными номерами неполных частных:
			\begin{center}	
				$(a) [0;3,1,2,3]$ $(b) [-1;2,1,6,1]$ $(c) [0;3,1,2,1,1,1]$
			\end{center}	 
		\item Вычислите подходящие дроби следующих цепных дробей:
		\begin{center}
			$(a) [1;2,3,3,2,1]$ $(b) [-3;1,1,1,1,3]$ $(c) [0;2,4,1,8,2]$
		\end{center}
	\item (a)Если $C_k=p_k/q_k$ является $k$-й правильной дробью простой цепной дроби $[1;2,3,4,...,n,n+1]$, покажите что:
	$$p_n=np_{n-1}+np_{n-2}+(n-1)p_{n-3}+...+3p_1+2p_0+(p_0+1)$$
	\begin{center}
		[\textit{Подсказка:} используйте соотношения: $p_0=1,p_1=3,p_k=(k+1)p_{k-1}+p_{k-2}$ для $k=2,...,n]$
		\end{center}
	(b)Покажите часть (a) при помощи вычисления числителя $p_4$ для $[1;2,3,4,5]$.
	\item Вычислите $p_k,q_k$ и $C_k(k=0,1,...,8)$ для простых цепных дробей приведенных ниже; следует принять к сведенью что подходящие дроби дают приближенные значения для чисел приведенных в скобках.
	$$(a) [1;2,2,2,2,2,2,2,2](\sqrt{2})$$
	$$(b) [1;1,2,1,2,1,2,1,2](\sqrt{3})$$
	$$(c) [1;4,4,4,4,4,4,4,4](\sqrt{5})$$
	$$(d) [1;2,4,2,4,2,4,2,4](\sqrt{6})$$
	$$(e) [1;1,1,1,4,1,1,1,4](\sqrt{7})$$
	\item Если $C_k=p_k/q_k$ - $k$-я правильная дробь простой непрерывной дроби $[a_0,a_1,...,a_n]$, подтвердите, что: 
		\begin{center}
	$q_k\ge2^{(k-1)/2}\qquad\qquad\qquad(2\le$k$\le$n$)$
		\end{center}
	[\textit{совет}: Рассматривайте $q_k=a_kq_{k-1}+q_{k-2}\ge2q_{k-2}$.]
	
	\item Найдите представление числа 3.1416 ввиде цепной дроби, и также для 3.14159.
	
	\item Если $C_k=p_k/q_k$ - к-я правильная дробь непрерывной дроби $[a_0,a_1,...,a_n]$ и $a_0>0$, покажите что:
	$$p_k/p_{k-1}=[a_k,a_{k-1},...,a_1,a_0]$$
	и
	$$q_k/q_{k-1}=[a_k;a_{k-1},...,a_2,a_1]$$
	[\textit{Совет}:В первом случае, примите что:
	$$p_k/p_{k-1}=a_k+(p_{k-2}/p_{k-1})=$$
	$$=a_k+\frac{1}{p_{k-1}/p_{k-2}}]$$
	\item Использую определение цепных дробей, определите решения следующих диофантовых уравнений:
	
	$(a)\qquad19x+51y=1;\qquad(b)\qquad364x+227y=1$
	
		$(c)\qquad18x+5y=1;\qquad(d)\qquad158x-57y=1$
		\\
\item Подтвердите Теорему 13-8 для цепной дроби $[1;1,1,1,1,1,1,1]$ 
\end{enumerate}

		
\section{Бесконечные цепные дроби}

До этого момента рассматривались лишь цепные дроби; и они, когда простые, являются рациональными числами. Одно из главных применений теории цепных дробей это примерное вычисления иррациональных чисел. Для этого необходимо понятие бесконечной цепной дроби.
		
Если $a_0,a_1,a_2,...$ - непрерывная последовательность простых чисел, все из которых положительны, кроме быть может $a_0$ тогда выражение: 
		
		$$a_0+\cfrac{1}{a_1+\cfrac{1}{a_2+\cfrac{1}{a_3+\ddots}}},$$
		
Или более простое обозначение $[a_0;a_1,a_2,...]$, называется бесконечной простой непрерывной дробью. Что бы дать этому определению математическое значение предположим что каждая из конечных цепных дробей
$$C_n=(a_0,a_1,...,a_n)$$
определена. Будет логично определить значение бесконечной цепной дроби $[a_0,a_1,a_2,...]$ как предел последовательности рациональных чисел $C_n$, конечно если данные предел сущевствует.Для упрощения обозначения, будем использовать $[a_0,a_1,a_2,...]$ для обозночения не только бесконечной цепной дроби но и ее обозначения.

Вопрос о существовании верхнего предела решается довольно легко.По гипотезе, данный предел не только существует, но и является иррациональным числом.Что бы увидеть это, рассмотрим приведенные выше формулы для конечных цепных дробей, который также подходят и для бесконечных цепных дробей, учитывая что вывод этих отношений не зависит от ограниченности дроби. Когда у индексов отсуствует верхний предел, теорема 13-8 утверждает что подходящие дроби $C_n$ из $[a_0,a_1,a_2,...]$ удовлетворяют бесконечной цепочке неравенств:
$$C_0<C_2<C_4<...<C_{2n}<...<C_{2n+1}<...<C_5<C_3<C_1$$
Учитывая что подходящие дроби с четными номерами $C_{2n}$ формируют монотонно возравастающую последовательность, ограниченную сверху $C_1$, они будут сходиться к пределу $\alpha$, который больше чем любая $C_{2n}$.Аналогично, монотонно убывающая последовательность подходящих дробей с нечетными номерами $C_{2n+1}$ ограничена снизу $C_0$ и имеет предел $\alpha'$, который меньше чем каждый из $C_{2n+1}$. Покажем что данные пределы эквивалентны. В основе отношения $p_{2n+1}q_{2n}-q_{2n+1}p_{2n}=(-1)^{2n}$ мы видим что:
$$\alpha'-\alpha<C_{2n+1}-C_{2n}=\frac{p_{2n+1}}{q_{2n+1}}-\frac{p_{2n}}{q_{2n}}$$
когда
$$0\ge|\alpha'-\alpha|<\frac{1}{q_{2n}q_{2n+1}}<\frac{1}{q_{2n}^2}$$
Учитывая что $q_i$ возрастает бесконечно с ростом $i$, правая часть этого неравенства может быть бесконечно малой. Если $\alpha'$ и $\alpha$ не одинаковы, то возникает противоречие(а именно, $1/q_{2n}^2 может быть меньше значения |\alpha'-\alpha|)$.Таким образом, две последовательности четных и нечетных подходящих дробей обладают одним ограничавющим значением $\alpha$, что означает - последовательность подходящих дробей $C_n$ обладает пределом $\alpha$.
Сделав вывод из этих заключений мы приходим к следующему определению:
\end{document}

